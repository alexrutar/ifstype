% header -----------------------------------------------------------------------
% Template created by texnew (author: Alex Rutar); info can be found at 'https://github.com/alexrutar/texnew'.
% version (1.13)


% doctype ----------------------------------------------------------------------
\documentclass[11pt, a4paper]{memoir}
\usepackage[utf8]{inputenc}
\usepackage[left=2cm,right=2cm,top=2cm,bottom=3cm]{geometry}
\usepackage[protrusion=true,expansion=true]{microtype}


% packages ---------------------------------------------------------------------
\usepackage{amsmath,amssymb,amsfonts}
\usepackage{graphicx}
\usepackage{etoolbox}

% Set enimitem
\usepackage{enumitem}
\SetEnumitemKey{nl}{nolistsep}
\SetEnumitemKey{r}{label=(\roman*)}

% Set tikz
\usepackage{tikz, pgfplots}
\pgfplotsset{compat=1.15}
\usetikzlibrary{intersections,positioning,cd}
\usetikzlibrary{arrows,arrows.meta}
\tikzcdset{arrow style=tikz,diagrams={>=stealth}}


% macros -----------------------------------------------------------------------
\DeclareMathOperator{\N}{{\mathbb{N}}}
\DeclareMathOperator{\Q}{{\mathbb{Q}}}
\DeclareMathOperator{\Z}{{\mathbb{Z}}}
\DeclareMathOperator{\R}{{\mathbb{R}}}
\DeclareMathOperator{\C}{{\mathbb{C}}}
\DeclareMathOperator{\F}{{\mathbb{F}}}

% Boldface includes math
\newcommand{\mbf}[1]{{\boldmath\bfseries #1}}

% proof implications
\newcommand{\imp}[2]{($#1\Rightarrow#2$)\hspace{0.2cm}}
\newcommand{\impe}[2]{($#1\Leftrightarrow#2$)\hspace{0.2cm}}
\newcommand{\impr}{{($\Rightarrow$)\hspace{0.2cm}}}
\newcommand{\impl}{{($\Leftarrow$)\hspace{0.2cm}}}

% align macros
\newcommand{\agspace}{\ensuremath{\phantom{--}}}
\newcommand{\agvdots}{\ensuremath{\hspace{0.16cm}\vdots}}

% convenient brackets
\newcommand{\brac}[1]{\ensuremath{\left\langle #1 \right\rangle}}
\newcommand{\norm}[1]{\ensuremath{\left\lVert#1\right\rVert}}
\newcommand{\abs}[1]{\ensuremath{\left\lvert#1\right\rvert}}

% arrows
\newcommand{\lto}[0]{\ensuremath{\longrightarrow}}
\newcommand{\fto}[1]{\ensuremath{\xrightarrow{\scriptstyle{#1}}}}
\newcommand{\hto}[0]{\ensuremath{\hookrightarrow}}
\newcommand{\mapsfrom}[0]{\mathrel{\reflectbox{\ensuremath{\mapsto}}}}
 
% Divides, Not Divides
\renewcommand{\div}{\bigm|}
\newcommand{\ndiv}{%
    \mathrel{\mkern.5mu % small adjustment
        % superimpose \nmid to \big|
        \ooalign{\hidewidth$\big|$\hidewidth\cr$/$\cr}%
    }%
}

% Convenient overline
\newcommand{\ol}[1]{\ensuremath{\overline{#1}}}

% Big \cdot
\makeatletter
\newcommand*\bigcdot{\mathpalette\bigcdot@{.5}}
\newcommand*\bigcdot@[2]{\mathbin{\vcenter{\hbox{\scalebox{#2}{$\m@th#1\bullet$}}}}}
\makeatother

% Big and small Disjoint union
\makeatletter
\providecommand*{\cupdot}{%
  \mathbin{%
    \mathpalette\@cupdot{}%
  }%
}
\newcommand*{\@cupdot}[2]{%
  \ooalign{%
    $\m@th#1\cup$\cr
    \sbox0{$#1\cup$}%
    \dimen@=\ht0 %
    \sbox0{$\m@th#1\cdot$}%
    \advance\dimen@ by -\ht0 %
    \dimen@=.5\dimen@
    \hidewidth\raise\dimen@\box0\hidewidth
  }%
}

\providecommand*{\bigcupdot}{%
  \mathop{%
    \vphantom{\bigcup}%
    \mathpalette\@bigcupdot{}%
  }%
}
\newcommand*{\@bigcupdot}[2]{%
  \ooalign{%
    $\m@th#1\bigcup$\cr
    \sbox0{$#1\bigcup$}%
    \dimen@=\ht0 %
    \advance\dimen@ by -\dp0 %
    \sbox0{\scalebox{2}{$\m@th#1\cdot$}}%
    \advance\dimen@ by -\ht0 %
    \dimen@=.5\dimen@
    \hidewidth\raise\dimen@\box0\hidewidth
  }%
}
\makeatother


% macros (theorem) -------------------------------------------------------------
\usepackage[hidelinks]{hyperref}
\usepackage[thmmarks,amsmath,hyperref]{ntheorem}
\usepackage[capitalise,nameinlink]{cleveref}

% Numbered Statements
\theoremstyle{change}
\theoremindent\parindent
\theorembodyfont{\itshape}
\theoremheaderfont{\bfseries\boldmath}
\newtheorem{theorem}{Theorem.}[section]
\newtheorem{lemma}[theorem]{Lemma.}
\newtheorem{corollary}[theorem]{Corollary.}
\newtheorem{proposition}[theorem]{Proposition.}

% Claim environment
\theoremstyle{plain}
\theorempreskip{0.2cm}
\theorempostskip{0.2cm}
\theoremheaderfont{\scshape}
\newtheorem{claim}{Claim}
\renewcommand\theclaim{\Roman{claim}}
\AtBeginEnvironment{theorem}{\setcounter{claim}{0}}

% Un-numbered Statements
\theorempreskip{0.1cm}
\theorempostskip{0.1cm}
\theoremindent0.0cm
\theoremstyle{nonumberplain}
\theorembodyfont{\upshape}
\theoremheaderfont{\bfseries\itshape}
\newtheorem{definition}{Definition.}
\theoremheaderfont{\itshape}
\newtheorem{example}{Example.}
\newtheorem{remark}{Remark.}

% Proof / solution environments
\theoremseparator{}
\theoremheaderfont{\hspace*{\parindent}\scshape}
\theoremsymbol{$//$}
\newtheorem{solution}{Sol'n}
\theoremsymbol{$\blacksquare$}
\theorempostskip{0.4cm}
\newtheorem{proof}{Proof}
\theoremsymbol{}
\newtheorem{nmproof}{Proof}

% Format references
\crefformat{equation}{(#2#1#3)}


% macros (algebra) -------------------------------------------------------------
\DeclareMathOperator{\Ann}{Ann}
\DeclareMathOperator{\Aut}{Aut}
\DeclareMathOperator{\chr}{char}
\DeclareMathOperator{\coker}{coker}
\DeclareMathOperator{\disc}{disc}
\DeclareMathOperator{\End}{End}
\DeclareMathOperator{\Fix}{Fix}
\DeclareMathOperator{\Frac}{Frac}
\DeclareMathOperator{\Gal}{Gal}
\DeclareMathOperator{\GL}{GL}
\DeclareMathOperator{\Hom}{Hom}
\DeclareMathOperator{\id}{id}
\DeclareMathOperator{\im}{im}
\DeclareMathOperator{\lcm}{lcm}
\DeclareMathOperator{\Nil}{Nil}
\DeclareMathOperator{\rank}{rank}
\DeclareMathOperator{\Res}{Res}
\DeclareMathOperator{\Spec}{Spec}
\DeclareMathOperator{\spn}{span}
\DeclareMathOperator{\Stab}{Stab}
\DeclareMathOperator{\Tor}{Tor}

% Lagrange symbol
\newcommand{\lgs}[2]{\ensuremath{\left(\frac{#1}{#2}\right)}}

% Quotient (larger in display mode)
\newcommand{\quot}[2]{\mathchoice{\left.\raisebox{0.14em}{$#1$}\middle/\raisebox{-0.14em}{$#2$}\right.}
                                 {\left.\raisebox{0.08em}{$#1$}\middle/\raisebox{-0.08em}{$#2$}\right.}
                                 {\left.\raisebox{0.03em}{$#1$}\middle/\raisebox{-0.03em}{$#2$}\right.}
                                 {\left.\raisebox{0em}{$#1$}\middle/\raisebox{0em}{$#2$}\right.}}


% macros (analysis) ------------------------------------------------------------
\DeclareMathOperator{\M}{{\mathcal{M}}}
\DeclareMathOperator{\B}{{\mathcal{B}}}
\DeclareMathOperator{\ps}{{\mathcal{P}}}
\DeclareMathOperator{\pr}{{\mathbb{P}}}
\DeclareMathOperator{\E}{{\mathbb{E}}}
\DeclareMathOperator{\supp}{supp}
\DeclareMathOperator{\sgn}{sgn}

\renewcommand{\Re}{\ensuremath{\operatorname{Re}}}
\renewcommand{\Im}{\ensuremath{\operatorname{Im}}}
\renewcommand{\d}[1]{\ensuremath{\operatorname{d}\!{#1}}}


% file-specific preamble -------------------------------------------------------
% REPLACE


% constants --------------------------------------------------------------------
\newcommand{\subject}{REPLACE}
\newcommand{\semester}{REPLACE}


% formatting -------------------------------------------------------------------
% Fonts
\usepackage{kpfonts}
\usepackage{dsfont}

% Equation numbering
\numberwithin{equation}{section}

% Footnote
\setfootins{0.5cm}{0.5cm} % footer space above
\renewcommand*{\thefootnote}{\fnsymbol{footnote}} % footnote symbol

% Table of Contents
\renewcommand{\thechapter}{\Roman{chapter}}
\counterwithout{section}{chapter}
\counterwithin*{chapter}{part}
\renewcommand*{\cftchaptername}{Chapter } % Place 'Chapter' before roman
\setlength\cftchapternumwidth{4em} % Add space before chapter name
\cftpagenumbersoff{chapter} % Turn off page numbers for chapter

% Section / Subsection headers
\newcommand*{\shortcenter}[1]{%
    \sethangfrom{\noindent ##1}%
    \Large\boldmath\scshape\bfseries
    \centering
\parbox{5in}{\centering #1}\par}
\setsecheadstyle{\shortcenter}
\setsubsecheadstyle{\large\scshape\boldmath\bfseries\raggedright}

% Chapter Headers
\chapterstyle{verville}

% Page Headers / Footers
\setsecnumdepth{subsection}
\copypagestyle{myruled}{ruled} % Draw formatting from existing 'ruled' style
\makeoddhead{myruled}{}{}{\scshape\subject}
\makeevenfoot{myruled}{}{\thepage}{}
\makeoddfoot{myruled}{}{\thepage}{}
\pagestyle{myruled}
\setfootins{0.5cm}{0.5cm}
\renewcommand*{\thefootnote}{\fnsymbol{footnote}}

% Titlepage
\title{\subject}
\author{Alex Rutar\thanks{\itshape arutar@uwaterloo.ca}\\ University of Waterloo}
\date{\semester\thanks{Last updated: \today}}

\begin{document}
\pagenumbering{gobble}
\hypersetup{pageanchor=false}
\maketitle
\newpage
\frontmatter
\hypersetup{pageanchor=true}
\tableofcontents*
\newpage
\mainmatter


% main document ----------------------------------------------------------------
% \subsection{Generations and Net Intervals}
% \begin{definition}
    % By a \textbf{generation rule} we mean a sequence of disjoint sets of words $\{\Lambda_n\}_{n=0}^\infty$ satisfying the following properties:
    % \begin{enumerate}[nl,r]
        % \item $\Lambda_0$ consists of exactly the empty word.
        % \item If $\sigma\in\Sigma^*$ is any infinite word, for each $n\geq 1$ there is a unique $i_n$ so that $(\sigma_1,\ldots,\sigma_{i_n})\in\Lambda_n$.
        % \item There is some constant $M>0$ (not dependent on $n$) such that for any words $\sigma,\tau\in\Lambda_n$
            % \begin{equation*}
                % |\log(|r_\sigma|)-\log(|r_\tau|)|<M
            % \end{equation*}
    % \end{enumerate}
% \end{definition}
% We say that a word $\sigma\in\Lambda_n$ is of \mbf{generation $n$}.
% As a consequence of (ii), each $\Lambda_n$ must be finite.
% \begin{example}
    % Set $r_{\min}=\min_j|r_j|$, $\Lambda_0$ the set consisting of the empty word, and define for $n\geq 1$
    % \begin{equation*}
        % \Lambda_n=\{\sigma\in\Sigma:|r_\sigma|\leq r_{\min}^n\text{ and }|r_{\sigma^i}|>r_{\min}^n\}
    % \end{equation*}
    % Taking $M=r_{\min}$, we see that $\{\Lambda_n\}_{n=0}^\infty$ is a generation rule.
% \end{example}
% Fix a generation rule $\{\Lambda_n\}_{n=0}^\infty$.
% For each positive integer $n$, let $h_1,\ldots,h_{s_n}$ be the collection of elements of the set $\{S_{\sigma(0)},S_{\sigma(1)}:\sigma\in\Lambda_n\}$ listed in increasing order.
% We set
% \begin{equation*}
    % \mathcal{F}_n=\{[h_j,h_{j+1}]:1\leq j\leq s_{n}-1\text{ and }(h_j,h_{j+1})\cap K\neq\emptyset\}
% \end{equation*}
% Elements of $\mathcal{F}_n$ are called \mbf{net intervals of generation $n$}.
% Let $[a,b]=\Delta\in\mathcal{F}_n$ and write $l(\Delta)=b-a$.
% \begin{definition}
    % A pair $(a,L)$ is a \textbf{neighbour} of $\Delta\in\mathcal{F}_n$ if there is some $\sigma\in\Lambda_n$ such that $S_\sigma(0,1)\cap\Delta\neq\emptyset$, $\ell(\Delta)^{-1}r_\sigma=L$, and $\ell(\Delta)^{-1}(a-S_\sigma(0))=a$.
    % Then the \textbf{neighbour set} of $\Delta$ is the ordered tuple
    % \begin{equation*}
        % V_n(\Delta)=((a_1,L_1),\ldots,(a_i,L_i))
    % \end{equation*}
    % where each $(a_i,L_i)$ is a neighbour of $\Delta$.
    % We order these tuples so that $a_i\leq a_{i+1}$ and if $a_i=a_{i+1}$, then $L_i<L_{i+1}$.
% \end{definition}
% \begin{definition}
    % We say that the IFS $\{S_i\}_{i=1}^\infty$ is \textbf{finite type} if there exists some generation rule $\{\Lambda_n\}_{n=0}^\infty$ such that there are only finitely many $V_n(\Delta)$.
% \end{definition}
% Given a net interval $\Delta\in\mathcal{F}_{n}$, there is a unique net interval $\Delta'\in\mathcal{F}_{n-1}$ such that $\Delta\subseteq\Delta'$.
% We then say that the \mbf{normalized length} of $\Delta$ is given by $\ell(\Delta)/\ell(\Delta')$.
% \begin{theorem}
    % The normalized length and neighbour set of any child of $\Delta$ depends only on the normalized length and neighbour set of $\Delta$.
% \end{theorem}
% \begin{proof}
    % If $\sigma=(i_1,\ldots,i_n)\in\Sigma^*$ is any word, for each $j=0,\ldots,k$, let $\sigma_j=(i_1,\ldots,i_n,j)$ so that by the defining identity of $\mu$, we have
    % \begin{align}
        % p_\sigma\mu\circ S_\sigma^{-1}(E) &= p_\sigma\sum_{j=0}^k p_j\mu(S_\sigma\circ S_j^{-1}(E))\nonumber\\
                                          % &= \sum_{j=0}^k p_{\sigma_j}\mu(S_{\sigma_j}^{-1}(E))\label{e:mu-2}
    % \end{align}
    % Then since $\{\Lambda_n\}_{n=0}^\infty$ is a generation rule, by property (i), for any finite word $\sigma$ not of the form $\sigma=\sigma_1\sigma_2$ with $\sigma_1\in\Lambda_n$, for each $j=0,\ldots,k$ there exists some (finite and possibly length 0 word) $\tau$ so that $\sigma_j\tau\in\Lambda_n$.
    % Thus by inductively applying \cref{e:mu-def} to words not yet of generation $n$, we have from \cref{e:mu-2}
    % \begin{equation}
        % \mu(E)=\sum_{\sigma\in\Lambda_n}p_\sigma\mu(S_\sigma^{-1}(E))
    % \end{equation}
    % for any $E\in\mathcal{B}([0,1])$.
% \end{proof}

% Comments: condition (iii) in generation rule is very sensitive.
% We want some sort of requirement that the net intervals of generation $n$ are close in length.
% Even if the words are of generation $n$ are close in size, it can happen that some net interval of generation $n$ is very small.
% \section{Weak Separation Property}
% Let
% \begin{align*}
    % \mathcal{I}_b&:= \{\sigma=(j_1,\ldots,j_n)\in\Sigma^*:r_\sigma\leq b<r_{\sigma^-}\}\\
    % \mathcal{A}_b&:= \{S_j:j\in\mathcal{I}_b\}
% \end{align*}
% We then have the following definition:
% \begin{definition}
    % An IFS $\{S_i\}_{i=1}^N$ has the weak separation property (WSP) if there exists some $x_0\in \R$ and $\ell\in\N$ such that for any $\sigma\in\Sigma^*$ and $0 < b < 1$, any closed ball with radius $b$ contains no more than distinct points of the form $S(S_\sigma (x_0 )),S\in \mathcal{A}_b$.
% \end{definition}
% In Lau, Ngai (2007), they define the notion of ``generalized finite type'' as follows

\chapter{Finite Type IV}
\section{Basic Definitions and Terminology}
\subsection{Iterated Function Systems}
Throughout, we let $\lambda$ denote the Lebesgue measure on $\R$.
\begin{definition}
    An \textbf{IFS} is a finite set of contractions
    \begin{equation*}
        S_i(x) = r_ix+a_i:\R\to\R\text{ for each }i=0,1,\ldots,k
    \end{equation*}
    with $k\geq 1$ and $0<|r_i|<1$.
\end{definition}
Each IFS generates a unique invariant compact set $K$, known as its associated \textbf{self-similar set}, satisfying
\begin{equation*}
    K=\bigcup_{j=0}^kS_j(K).
\end{equation*}
Rescaling the $r_i$ and $a_i$ if necessary, we may assume the convex hull of $K$ is $[0,1]$.
We also associate probabilities $0<p_i<1$ for each $i=0,\ldots,k$ which satisfy $\sum_i p_i=1$.
To these probabilities there is a unique self-similar measure $\mu$ with $\supp\mu=K$ satisfying
\begin{equation}
    \mu=\sum_{j=0}^k p_j\mu\circ S_j^{-1}.\label{e:mu-def}
\end{equation}
We are primarily interested in studying properties of this measure $\mu$.

Let $\Sigma=\{0,1,\ldots,k\}$ be our alphabet, let $\Sigma^k$ denote the words of length $k$, and $\Sigma^*=\bigcup_{k=0}^\infty\Sigma_k$ denote the set of all the finite words on $\Sigma$.
Given $\sigma=(\sigma_1,\ldots,\sigma_j)\in\Sigma$, we let
\begin{align*}
    \sigma^- &=(\sigma_1,\ldots,\sigma_{j-1})\\
    S_\sigma &= S_{\sigma_1}\circ\cdots\circ S_{\sigma_j}\\
\end{align*}
and similarly,
\begin{align*}
    r_\sigma = \prod_{i=1}^j r_{\sigma_i}\text{ and }p_\sigma = \prod_{i=1}^j p_{\sigma_i}.
\end{align*}
Additionally, we set $r_{\max}=\max_{i}|r_i|$.

\subsection{Generations}
For any $0<\alpha\leq 1$, we define the family
\begin{equation*}
    \Lambda_\alpha=\{\sigma\in\Sigma^*:|r_\sigma|<\alpha\leq|r_{\sigma^-}|\}
\end{equation*}
called the \mbf{words of generation $\alpha$}.
Given a word $\sigma$, we say that the \textbf{generation} $G(\sigma)$ is the interval $(|r_\sigma|,|r_{\sigma^-}|]$.
By definition $\alpha\in G(\sigma)$ if and only if $\sigma\in\Lambda_\alpha$.

Let $h_1,\ldots,h_{s(\alpha)}$ be the collection of elements of the set $\{S_{\sigma(0)},S_{\sigma(1)}:\sigma\in\Lambda_\alpha\}$ listed in increasing order.
We set
\begin{equation*}
    \mathcal{F}_\alpha=\{[h_j,h_{j+1}]:1\leq j\leq s(\alpha)-1\text{ and }(h_j,h_{j+1})\cap K\neq\emptyset\}
\end{equation*}
Elements of $\mathcal{F}_\alpha$ are called \mbf{net intervals of generation $\alpha$}.
\begin{definition}
    A pair $(a,L)$ is a \textbf{neighbour} of $\Delta=[a_0,b_0]\in\mathcal{F}_n$ if there is some $\sigma\in\Lambda_\alpha$ such that $S_\sigma(0,1)\cap\Delta\neq\emptyset$, $\lambda(\Delta)^{-1}r_\sigma=L$, and $\lambda(\Delta)^{-1}(a_0-S_\sigma(0))=a$, and we say that $\sigma$ \textbf{generates} the neighbour $(a,L)$.
    Then the \textbf{neighbour set} of $\Delta$ is the ordered tuple
    \begin{equation*}
        V_\alpha(\Delta)=((a_1,L_1),\ldots,(a_j,L_j))
    \end{equation*}
    where each $(a_i,L_i)$ is a (distinct) neighbour of $\Delta$.
    We order these tuples so that $a_i\leq a_{i+1}$ and if $a_i=a_{i+1}$, then $L_i<L_{i+1}$.
\end{definition}
Abusing notation slightly, we say that $\mathcal{F}_\alpha\neq\mathcal{F}_\beta$ if either the net intervals are distinct, or if they are the same, then the neighbour set of some net interval is different.
(TODO: is this actually abusing notation? Or are the notions equivalent?)
We can generalize the notion of generation to net intervals.
Let $\Delta\in\mathcal{F}_\alpha$ have neighbour set $((a_1,L_1),\ldots,(a_j,L_j))$.
For each $i$, let $\sigma_i$ generate $(a_i,L_i)$.
Then we call
\begin{equation*}
    G(\Delta)=\bigcap_{i=1}^j G(\sigma_i)
\end{equation*}
the \textbf{generation} of some $\Delta\in\mathcal{F}_\alpha$.
Note that
\begin{enumerate}[nl,r]
    \item $\alpha\in G(\Delta)$
    \item For any $\beta\in G(\Delta)$, $\Delta\in\mathcal{F}_\beta$ and $V_\beta(\Delta)=V_\alpha(\Delta)$
    \item If $\gamma\notin G(\Delta)$, either $\Delta\notin\mathcal{F}_\gamma$ or $V_\gamma(\Delta)\neq V_\alpha(\Delta)$.
\end{enumerate}

\subsection{Transition Types}
Let $0<\alpha\leq 1$ and $\Delta\in\mathcal{F}_\alpha$ and suppose $\Delta$ has neighbour set $((a_1,L_1),\ldots,(a_n,L_n))$
Set
\begin{equation*}
    L_{\max}=\max\{|L_i|:i=1,\ldots,n\}\text{ and }\gamma = \lambda(\Delta)\cdot L_{\max};
\end{equation*}
in other words, $\gamma$ is the largest value achieved by $|r_\sigma|$ where $\sigma$ generates some neighbour of $\Delta$.
Let $\Delta$ have children $(\Delta_1,\ldots,\Delta_n)\in\mathcal{F}_\gamma$.
As in the proof of \cref{p:gfin}, we see that either $n>1$ or if $n=1$, then $V_\alpha(\Delta)\neq V_\gamma(\Delta_1)$.
We call the tuple $(\Delta_1,\ldots,\Delta_n)$ the \textbf{children} of $\Delta\in\mathcal{F}_\alpha$.
Note that it suffices to take any $\gamma'$ such that
\begin{equation*}
    \min\bigl\{r_{\max}\gamma,\lambda(\Delta)\cdot\max\{|L_i|:i=1,\ldots,n; |L_i|\neq L_{\max}\}\bigr\}<\gamma'\leq\gamma
\end{equation*}
where the inner maximum is taken to be 0 if the set is empty.
\begin{definition}
    Suppose $\Delta=[a,b]\in\mathcal{F}_\alpha$ has children $(\Delta_1,\ldots,\Delta_n)$ in generation $\gamma$.
    Write $\Delta_i=[a_i,a_i+L_i]$.
    We define the \textbf{transition type} of $\Delta$, denoted $\mathcal{C}_\alpha(\Delta)$, to be the tuple
    \begin{equation*}
        \left(\bigl(\frac{a_1-a}{\lambda(\Delta)},\frac{L_1}{\lambda(\Delta)},V_\gamma(\Delta_1)\bigr),\ldots,\bigl(\frac{a_n-a}{\lambda(\Delta)},\frac{L_n}{\lambda(\Delta)},V_\gamma(\Delta_n)\bigr)\right)
    \end{equation*}
\end{definition}
\begin{remark}
    To compute the children, it is not sufficient to consider $\gamma=|r_i|\alpha$ for some $0\leq i \leq k$.
    For example, in the main IFS example, take $\Delta=[4/15, 1/3]\in\mathcal{F}_\alpha$ where $\alpha=1/3$.
    Now $\Delta\in \mathcal{F}_{1/5}$, but has different neighbour set, and $\Delta\notin\mathcal{F}_{1/9}$ at all.
    However, the largest possible value of $|r_i|$ misses the children.
\end{remark}
\subsection{Basic Properties of IFS}
Here are some properties of the generations $\{\Lambda_\alpha\}_{0<\alpha\leq 1}$.
\begin{proposition}\label{p:gfin}
    Let $0<\alpha<\beta\leq 1$.
    \begin{enumerate}[nl,r]
        \item $\mathcal{F}_\alpha\neq\mathcal{F}_\beta$ if and only if there exists some $a_i\in\{0\}\cup\N$ such that $\rho:=\prod_{i=0}^k |r_i|^{a_i}$ satisfies $\alpha<\rho\leq\beta$.
        \item If $\sigma$ is any infinite word, there exists a unique index $i_\alpha$ such that $(\sigma_1,\ldots,\sigma_{i_\alpha})\in\Lambda_\alpha$.
            In particular, when $\alpha<\beta$, $i_\alpha\geq i_\beta$.
    \end{enumerate}
\end{proposition}
\begin{proof}
    Recall that $\sigma\in\Lambda_\alpha$ for any $\alpha\in(|r_\sigma|,|r_{\sigma^-}|]$.
    This is sufficient for the forward implication of (i) and (ii).

    To see the reverse implication of (i), suppose such a $\rho$ exists and let $\omega$ be a word such that $|r_\omega|=\rho$.
    Let $\sigma$ be a prefix of $\omega$ such that $\sigma\in\Lambda_\beta$; then since $\omega\notin\Lambda_\alpha$, $\sigma\notin\Lambda_\alpha$ as well.
    Let $\Delta\subseteq S_\sigma[0,1]$ with $\Delta\in\mathcal{F}_\alpha$ be any net interval, so that $\sigma$ generates some neighbour $(a,L)$ of $\Delta$ where $L=r_\sigma \lambda(\Delta)^{-1}$.
    If $\Delta\notin\mathcal{F}_\beta$, we are done, so let's suppose $\Delta=\Delta'\in\mathcal{F}_\beta$.
    Now suppose $(a',L')$ is any neighbour of $\Delta'$ generated by $\tau\in\Lambda_\beta$; it suffices to show that $(a',L')\neq(a,L)$.
    Suppose $a=a'$; then if $\tau$ generates $(a',L')$, we have $\sigma\precurlyleq\tau$, and since $\sigma\in\Lambda_\alpha$ and $\tau\in\Lambda_\beta$ with $\Lambda_\alpha\neq\Lambda_\beta$, we have $r_\sigma\neq r_\tau$ so $L\neq L'$.
\end{proof}
\begin{remark}
    One way to think about the children of an interval as follows.
    Enumerate the points $\left\{\prod_{i=0}^k|r_i|^{a_i}:a_i\in\{0\}\cup\N\right\}$ in decreasing order $(\rho_i)_{i=1}^\infty$.
    As in \cref{p:gfin}, the $\mathcal{F}_\alpha$ change on transitions of intervals $[\rho_{i+1},\rho_{i})$.
    However, if $\Delta\in\mathcal{F}_\alpha$ with $\alpha\in[\rho_{k+1},\rho_k)$, it may be that $\Delta\in \mathcal{F}_\beta$ for any $\beta\in[\rho_{k+2},\rho_{k+1})$, with $V_\beta(\Delta)=V_\alpha(\Delta)$.
    The children are the net intervals in generation $\rho_{m}$ where $m> k+1$ is minimal such that either $\Delta\notin\mathcal{F}_{\rho_m}$ or $V_{\rho_m}(\Delta)\neq V_\alpha(\Delta)$.
\end{remark}
\begin{proposition}\label{p:ttype}
    Consider the IFS $\{S_i\}_{i=0}^m$ and let $0<\alpha\leq 1$.
    Then $\mathcal{C}_\alpha(\Delta)$ depends only on the neighbour set $V_\alpha(\Delta)$.
\end{proposition}
\begin{proof}
    \textit{[Sketch.]}
    Let $\Delta\in \mathcal{F}_\alpha$ have neighbour set $\bigl((a_1,L_1),\ldots,(a_n,L_n)\bigr)$ and let $\{i_1,\ldots,i_m\}$ be the set of indices such that $L_{i_1}=\cdots=L_{i_m}=:L$ are maximal.
    Let $\gamma$ be such that $(\Delta_1,\ldots,\Delta_n)$ with $\Delta_i\in\mathcal{F}_\gamma$ are the children of $\Delta$.
    Let $\Gamma\subseteq\Lambda_\alpha$ be the set of words which generate some $(a_{i_j},L_{i_j})$ for $1\leq j\leq m$.
    Note the following facts:
    \begin{itemize}[nl]
        \item If $\sigma\in\Lambda_\alpha\setminus\Gamma$ has $S_\sigma[0,1]\supseteq\Delta$, then $\sigma\in\Lambda_\gamma$
        \item If $\sigma\in\Gamma$, then $\sigma\notin\Lambda_\gamma$ but $\sigma l\in\Lambda_\gamma$ for any $0\leq l\leq m$.
    \end{itemize}
    But then the words $\tau\in\Lambda_\gamma$ such that $S_\tau[0,1]\supseteq\Delta$ are precisely the words
    \begin{equation*}
        \tau \in\Lambda_\alpha\setminus\Gamma\text{ with } S_\tau[0,1]\supseteq\Delta
    \end{equation*}
    or
    \begin{equation*}
        \tau=\sigma l\text{ with }\sigma\in\Gamma\text{ and }l\in\{0,\ldots,k\}.
    \end{equation*}
    Thus the set $\{\tau\in\Lambda_\gamma:S_\tau[0,1]\supseteq\Delta\}$ depends only on $V_\alpha(\Delta)$, which fully determines $C_\alpha(\Delta)$.
\end{proof}
\subsection{Iterated Function Systems of Finite Type}
\begin{definition}
    We say that the IFS $\{S_i\}_{i=0}^k$ is \textbf{finite type} if there are only finitely many neighbourhood sets.
\end{definition}
\begin{example}
    The governing example throughout this section is the IFS given by
    \begin{align*}
        S_0(x)&=\frac{1}{3}x & S_1(x) &= \frac{1}{5}x+\frac{4}{15}\\
        S_2(x) &= \frac{1}{3}x+\frac{7}{15} & S_3(x) &= \frac{1}{5}x+\frac{4}{5}
    \end{align*}
    which is perhaps better summarized by the diagram $S_i[0,1]$ for $i=0,\ldots,3$:
    \begin{center}
        \begin{tikzpicture}[xscale=14,yscale=0.3]
            \foreach \x/\y/\ctr/\lep/\rep in {0/.333333333/0/0/\frac{1}{3},.466666667/.8/2/\frac{7}{15}/\frac{4}{5},.8/1/3/\,/1} {
                \draw[thick] (\x,1) -- node[above]{$S_{\ctr}[0,1]$}(\y,1);
                \draw[thick] (\x,1-0.2) -- (\x,1+0.2) node[above,scale=0.7]{$\lep$};
                \draw[thick] (\y,1-0.2) -- (\y,1+0.2) node[above,scale=0.7]{$\rep$};
            }
            \foreach \x/\y/\ctr/\lep/\rep in {.266666667/.466666667/1/\frac{4}{15}/\,} {
                \draw[thick] (\x,0) -- node[below]{$S_{\ctr}[0,1]$} (\y,0);
                \draw[thick] (\x,-0.2) node[below,scale=0.7]{$\lep$} -- (\x,0.2);
                \draw[thick] (\y,-0.2) node[below,scale=0.7]{$\rep$} -- (\y,0.2);
            }
        \end{tikzpicture}
    \end{center}
\end{example}
\begin{corollary}
    $\{S_i\}_{i=0}^k$ is finite type if and only if there are finitely many transition types.
\end{corollary}
\begin{proof}
    Follows from \cref{p:ttype}.
\end{proof}
\begin{remark}
    Computationally, one can prove that an IFS is of finite type as follows.
    Starting with $\Delta=[0,1]$, the children $(\Delta_1,\ldots,\Delta_n)$ in generation $\gamma$ and their neighbour sets $V_\gamma(\Delta_1),\ldots,V_\gamma(\Delta_n)$.
    Recursively repeat the process for any $\Delta_i$ in which $V_\gamma(\Delta_i)$ has not yet been observed.
    If this process terminates, then the IFS is of finite type.
\end{remark}
\begin{proposition}[Bounds on Interval Width]
    Let $\{S_i\}_{i=0}^m$ be an IFS of finite type.
    \begin{enumerate}[nl,r]
        \item There exists a constant $M\geq 1$ such that for any $0<\alpha\leq 1$ and $\Delta\in\mathcal{F}_\alpha$, \begin{equation*} \frac{1}{M}\leq \frac{\lambda(\Delta)}{\alpha}\leq 1 \end{equation*}
        \item There exists a constant $M\geq 1$ such that for any $0<\alpha\leq 1$, for any $\Delta\in\mathcal{F}_\alpha$ with neighbours $\Delta^-$ and $\Delta^+$,
            \begin{equation*}
                \frac{1}{M}\leq \frac{\lambda(\Delta)}{\lambda(\Delta^-)}\leq M\text{ and } \frac{1}{M}\leq \frac{\lambda(\Delta)}{\lambda(\Delta^+)}\leq M.
            \end{equation*}
    \end{enumerate}
\end{proposition}
\begin{proof}
    \begin{enumerate}[nl,r]
        \item Let $S\subseteq\Lambda_\alpha$ denote the set of words such that for any $\sigma\in S$, $\Delta\subseteq S_\sigma[0,1]$.
            By definition of finite type, $\lambda(\Delta)^{-1}r_\sigma$ takes one of finitely many values $L_i$; let $\lambda(\Delta)^{-1}r_\omega=L$ denote the maximum of such values.
            Then $\lambda(\Delta)\geq r_\omega/L$, so that
            \begin{equation*}
                \frac{1}{r_{\max}L}\leq \frac{r_\omega}{L\alpha}\leq \frac{\lambda(\Delta)}{\alpha}
            \end{equation*}
            where $r_{\max}=\max_{0\leq i\leq m}|r_i|$.
            The upper inequality follows since $r_\sigma/\alpha\leq 1$ for any $\sigma\in\Lambda_\alpha$.
        \item Immediate from (i).
    \end{enumerate}
\end{proof}
\section{Transition Matrices}
Let $0<\alpha<\beta\leq 1$ be arbitrary and suppose $\Delta=[a,b]\in\mathcal{F}_\alpha$ is arbitrary.
Let $\widehat\Delta=[c,d]\in\mathcal{F}_\beta$ be the parent of $\Delta$, and suppose
\begin{align*}
    V_\alpha(\Delta) &= ((a_1,L_1),\ldots,(a_I,L_I))\\
    V_\beta(\Delta) &= ((c_1,M_1),\ldots,(c_I,M_J))
\end{align*}
Then the \textbf{transition matrix} $T_{\beta\to\alpha}(\Delta)$ is the $I\times J$ matrix defined as follows for fixed $i,j$.
Let $\sigma\in\Lambda_\beta$ such that $S_\sigma(x)=\lambda(\widehat\Delta)\cdot (M_jx-c_j)+c$.
Then $T_{ij}=\sum_{\omega\in S}p_\omega$ where
\begin{equation*}
    S=\{\omega:\sigma\omega\in\Lambda_\alpha\text{ and }S_{\sigma\omega}(x)=\lambda(\Delta)\cdot(L_ix-a_i)+a\}
\end{equation*}
and the empty sum is understood to be 0.
It is straightforward to see that transition matrices are well-defined, since if $\sigma$ and $\sigma'$ satisfy $S_\sigma(x)=S_{\sigma'}(x)$, then $S_{\sigma\omega}(x)=S_{\sigma'\omega}(x)$ for any word $\omega$.
\begin{proposition}[Properties of Transition Matrices]
    The following hold:
    \begin{enumerate}[nl,r]
        \item Suppose $0<\alpha<\gamma<\beta\leq 1$ and $\Delta\in\mathcal{F}_\alpha$ has parent $\widehat\Delta\in\mathcal{F}_\gamma$.
            Then $T_{\beta\to\alpha}(\Delta)=T_{\beta\to\gamma}(\widehat\Delta)\cdot T_{\gamma\to\alpha}(\Delta)$.
    \end{enumerate}
\end{proposition}
\section{Misc ideas}
\begin{proposition}
    Let $0<\alpha\leq 1$ and $\Delta\in\mathcal{F}_\alpha$ arbitrary.
    Let $r=|r_i|$ for some $i$, so that $\Delta$ has children $(\Delta_1,\ldots,\Delta_n)$ ordered left to right in $\mathcal{F}_{r\alpha}$.
    Then the sequence $(\mu(\Delta_i),V_{r\alpha}(\Delta_i))_{i=1}^n$ depends only on $\alpha$, $\lambda(\Delta)$ and $V_\alpha(\Delta)$.
\end{proposition}
\begin{remark}
    One might hope to drop the requirement on knowing $\alpha$; however, this does not hold if so.
    However, this is may be sufficient to prove that there are only finitely many types:
\end{remark}

How hard to characterize transition maps $\Lambda_\alpha\to\Lambda_{r_i\alpha}$?
If there are only finitely many such transitions, we have finiteness requirement.

May need to do some sort of local argument?
Fix a value of $x$, and consider a monotonic sequence $(\alpha_i)_{i=1}^\infty$ with $0<\alpha_i<1$ and $(\alpha_i)\to 0$.
Then look at the sequence of character.

How hard to prove that something is actually finite type?
\end{document}

